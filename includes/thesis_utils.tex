% %%%%%%%%%%%%%%%%%%%%%%%%%%%%%%%%%%%%
% Commands for thesis
% %%%%%%%%%%%%%%%%%%%%%%%%%%%%%%%%%%%%

% ==============================================
% Title Page
% ==============================================
\usepackage{geometry}
\usepackage{afterpage}
\usepackage{pagecolor}
\usetikzlibrary{calc}

% TOC
\usepackage{fancyhdr}


% =====================================
% Chapters
% =====================================

\usepackage{epigraph}

\usepackage{titletoc}

\definecolor{gainsboro}{rgb}{0.86, 0.86, 0.86}
\definecolor{lavender_floral}{rgb}{0.71, 0.49, 0.86}
\definecolor{gray_asparagus}{rgb}{0.27, 0.35, 0.27}

\titlecontents*{chapter}
  [0pt]% <left>
  {}
  {\bf \chaptername\ \thecontentslabel:\quad}
  {}
  {\bfseries\hfill\contentspage}

\usepackage[calcwidth]{titlesec}
\usepackage{microtype}
\usepackage{tikz}


\colorlet{chapbgcolor}{gray_asparagus!50} % shaded background color for chapters
\colorlet{chapnumcolor}{black!55!gray_asparagus}% color for numbers in chapters

% from https://tex.stackexchange.com/a/233830/121799
\newcommand{\chaptitle}[1]{%
\begin{tikzpicture}
    % Box
    \fill[chapbgcolor!70,rounded corners=5pt] (-.1,2.5) rectangle (1.1\linewidth,0);
    % Text
    \node[align=right,anchor=east,inner sep=6pt,font=\huge\bfseries\sffamily] at (1.05\linewidth,1.25) {#1};
    % Number
    \node[font=\fontsize{60}{62}\usefont{OT1}{ptm}{m}{n}
        \selectfont\itshape\bfseries,text=chapnumcolor] at
    % (\linewidth,2.5)  % default
    (.065\linewidth, 1.25)
    {\thechapter};
\end{tikzpicture}
}

\newcommand{\chaptitlenonumber}[1]{%
\begin{tikzpicture}
    \fill[chapbgcolor!70,rounded corners=5pt] (0,2.5) rectangle (1.0\linewidth,0);
    \node[align=left,anchor=west,inner sep=6pt,font=\Huge\bfseries\sffamily] at (.15\linewidth,1.25) {#1};
\end{tikzpicture}
}

% No newpage between chapters
\makeatletter
\patchcmd{\chapter}{\if@openright\cleardoublepage\else\clearpage\fi}{}{}{}
\makeatother


% =====================================
% Other QCD
% =====================================
\newcommand{\alphas}{\ensuremath{\alpha_{\mathrm{s}}}\xspace}
\newcommand{\as}{\ensuremath{a_{\mathrm{s}}}\xspace}
\newcommand{\LambdaQCD}{\ensuremath{\Lambda_{\mathrm{QCD}}}\xspace}

% =====================================
% Chapter Appendices
% =====================================
\usepackage{appendix}
\usepackage{chngcntr}

% Tools for adding to TOC
\newcommand{\addnewlinetotoc}{
    \addtocontents{toc}{%
        % With hyperref, contentsline has 4 arguments:
        \protect\contentsline{section}{\\}{}{}
        % Without, it has 3, but I think using empty braces 4th should not affect anything even if we don't use hyperref
    }
}

% Sub-Appendix Header
\AtBeginEnvironment{subappendices}{%
    % Chapter titles without numbers
    \titleformat{\chapter}[display]
    {\normalfont\huge\bfseries\sffamily}{}{25pt}{\chaptitlenonumber}
    \titlespacing*{\chapter} {0pt}{110pt}{20pt}
    % Chapter title
    \clearpage
    \chapter*{Appendices for Chapter \thechapter}
    % phantomsection for use with hyperref, comment otherwise
    \phantomsection
    %
    \addcontentsline{toc}{chapter}
        {\bf\\Appendices for Chapter \thechapter}
    \counterwithin{figure}{section}
    \counterwithin{table}{section}
    % Chapter titles with numbers
    \titleformat{\chapter}[display]
    {\normalfont\huge\bfseries\sffamily}{}{25pt}{\chaptitle}
    \titlespacing*{\chapter} {0pt}{110pt}{20pt}
}


% =====================================
% Figures
% =====================================
% ---------------------------------
% Re-using "Picture Book" figures
% ---------------------------------
% See https://tex.stackexchange.com/a/225075
\NewEnviron{sourcefigure}[1][htbp]{%
    {\let\caption\relax\let\ref\relax
     \renewcommand{\label}[1]{%
         \gdef\sfname{sf:##1}%
     }%
     \setbox1=\hbox{\BODY}%
    }% Capture \label
    \global\expandafter\let\csname\sfname\endcsname\BODY% Capture entire figure
    \begin{figure}[#1]
        \BODY{}
    \end{figure}
}

\newcommand{\reusefigure}[2][htbp]{%
    {\addtocounter{figure}{-1}%
     \renewcommand{\theHfigure}{dupe-fig}  % For use with hyperref
     % \renewcommand{\thefigure}{\ref{fig:#2}}% Figure counter is \ref
     \renewcommand{\thefigure}{\ref{fig:#2} (repeated)}% Figure counter + "(repeated)"
     \renewcommand{\addcontentsline}[3]{}% Avoid placing figure in LoF
     \renewcommand{\label}[1]{}% Make \label inactive
     \begin{figure}[#1]
        \csname sf:fig:#2\endcsname
    \end{figure}%
    }
}


% =====================================
% Problems
% =====================================
\thmboxstyle{notebox}{mdorangebox}{orange!50!brown}{yellow!5!olive!5}

\newtheoremstyle{probstyle}%
{}{}%
{\upshape}{}%
{}{}%
{ }%
{\textcolor{orange!50!brown}{\textbf{\thmname{#1}\thmnumber{ #2:}}}}

\declaretheorem[style=probstyle,
                mdframed={linecolor=orange!50!brown,
                          backgroundcolor=yellow!5!olive!5,
                      },
                name=Problem,
                numberwithin=chapter]{problem}

\declaretheorem[style=probstyle,
                mdframed={linecolor=orange!50!brown,
                          backgroundcolor=yellow!5!olive!5,
                      },
                name=Bonus Problem]{bonusproblem}

\newcommand{\makeprob}[3]{
    \medskip
    \begin{problem}
        \textcolor{orange!50!brown}{\textbf{#1}}
        \\
        #3
        \ifstrempty{#2}%
            {}
            {\labsoln{#2}}
    \end{problem}
}

\newcommand{\makebonusprob}[3]{
    \medskip
    \begin{bonusproblem}
        \textcolor{orange!50!brown}{\textbf{#1}}
        \\
        #3
        \ifstrempty{#2}%
            {}
            {\labsoln{#2}}
    \end{bonusproblem}
}

% Labels with Links to Solutions
\DeclareRobustCommand{\labsoln}[1]{%
    \label{prob:#1}%
    \hyperref[soln:#1]{\texttt{(Solution)}}
}

% Solutions themselves
\DeclareRobustCommand{\Soln}[1]{%
    % phantomsection for use with hyperref, comment otherwise
    \phantomsection
    %
    \addcontentsline{toc}{subsection}{Problem~\ref{prob:#1}}
    \subsubsection*{\hyperref[prob:#1]{Problem~\ref{prob:#1}}}
    \label{soln:#1}
}


\DeclareRobustCommand{\BonusSoln}[1]{%
    % phantomsection for use with hyperref, comment otherwise
    \phantomsection
    %
    \addcontentsline{toc}{subsection}{Bonus Problem~\ref{prob:#1}}
    \subsubsection*{\hyperref[prob:#1]{Bonus Problem~\ref{prob:#1}}}
    \label{soln:#1}
}



\newenvironment{problems}{}{}
\AtBeginEnvironment{problems}{%
    % Chapter titles without numbers
    \titleformat{\chapter}[display]
    {\normalfont\huge\bfseries\sffamily}{}{25pt}{\chaptitlenonumber}
    \titlespacing*{\chapter} {0pt}{110pt}{20pt}
    % Chapter title
    \clearpage
    \chapter*{Problems for Chapter \thechapter}
    % phantomsection for use with hyperref, comment otherwise
    \phantomsection
    %
    \addcontentsline{toc}{chapter}
        {\bf\\Problems for Chapter \thechapter}
    \counterwithin{figure}{section}
    \counterwithin{table}{section}
    % Chapter titles with numbers
    \titleformat{\chapter}[display]
    {\normalfont\huge\bfseries\sffamily}{}{25pt}{\chaptitle}
    \titlespacing*{\chapter} {0pt}{110pt}{20pt}
}
\AtEndEnvironment{problems}{%
    \addnewlinetotoc
    \clearpage
}


\newcommand{\mellinconvolution}{\ensuremath{\star_{\mathcal M}}}



% =====================
% Specific Commands for PIRANHA
% =====================
\usepackage{hhline}
\newcommand{\zcut}{\ensuremath{z_{\rm cut}}}
\newcommand{\zcrit}{\ensuremath{z_{\rm crit}}}
\newcommand{\thetacrit}{\ensuremath{\theta_{\rm crit}}}
\newcommand{\zpre}{\ensuremath{z_{\rm pre}}}

\newcommand{\rhoC}{\ensuremath{\rho\,\mathcal{C}}}
\newcommand{\rhoU}{\ensuremath{\rho\,\mathcal{U}}}

\newcommand{\PIRANHA}{\textsc{Piranha}}

\newcommand{\PRSF}[1]{P-RSF\(_{f = #1}\)}
\newcommand{\SD}[1]{SD\(_{\beta = #1}\)}

% -----------------------------
% Custom colors
% -----------------------------
% Toy event visualization
\definecolor{Epluscolor}{rgb}{0.0, 0.62, 0.38}
% Pistachio

\definecolor{Eminuscolor}{rgb}{0.76, 0.23, 0.13}
% darkpastelred

\definecolor{E1color}{rgb}{0.76, 0.13, 0.28}

\definecolor{E2color}{rgb}{0.2, 0.2, 0.6}

% Groomers
\definecolor{softdrop}{rgb}{0.36, 0.54, 0.66}
\definecolor{rss}{rgb}{1.0, 0.49, 0.0}
