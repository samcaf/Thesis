%%%%%%%%%%%%%%%%%%%%%%%%%%
% Main Utilities:
%%%%%%%%%%%%%%%%%%%%%%%%%%
% Preamble for physics-paper-related packages and commands

% ========================
% Main packages
% ========================
\usepackage[utf8]{inputenc}

\usepackage{mathtools,amssymb,amsmath,amsthm,mathrsfs}
\usepackage{physics}
\usepackage{enumitem}
\usepackage{etoolbox}

% =====================================
% Hyperlinks
% =====================================
\usepackage{hyperref}

% ---------------------------------
% hyperref fixes
% ---------------------------------
% Sometimes hyperref will lead to problems, and these are some fixes:

% - - - - - - - - - - - - - - - - -
% Conflicts with \MakeUppercase
% https://tex.stackexchange.com/a/598876
% - - - - - - - - - - - - - - - - -
\def\MakeUppercaseUnsupportedInPdfStrings{\scshape}

% - - - - - - - - - - - - - - - - -
% Conflicts with... label?
% https://tex.stackexchange.com/a/130323
% - - - - - - - - - - - - - - - - -
\newcounter{numrel}% Counter for numering relations
\newcounter{Hnumrel}% Keep hyperref happy and don't duplicate anchors
\renewcommand{\thenumrel}{\roman{numrel}}% Counter numrel uses lowercase roman numerals

\makeatletter
\newcommand{\numrel}[2]{% Relation numbering
  \refstepcounter{numrel}% Increment numrel counter and create correct reference hook
  \stepcounter{Hnumrel}%
  \ifmeasuring@\else\ltx@label{#2}\fi % Label numrel counter (issue only once)
  \overset{\text{(\thenumrel)}}{#1}% Print counter + relation
}
\makeatother
\AfterEndEnvironment{align*}{\setcounter{numrel}{0}}% Resets numrel at the end of align*
% - - - - - - - - - - - - - - - - -


% ---------------------------------
% hyperref dummy functions
% ---------------------------------
% Adding dummy versions of hyperref tools to avoid errors if hyperref is _not_ used
\providecommand\phantomsection{}
\providecommand\theHfigure{}


% =====================================
% Glossary
% =====================================
% See https://en.wikibooks.org/wiki/LaTeX/Glossary
% Note: glossaries must be loaded after hyperref

% DEBUG: Not sure which options to use for glossaries (esp if I want an index)
% \usepackage[acronym,automake=immediate,nonumberlist,toc,xindy]{glossaries}
% \usepackage[acronym,nomain,toc,automake]{glossaries-extra}
% \usepackage[acronym,automake,toc,xindy]{glossaries-extra}
% \usepackage[acronym,automake,toc]{glossaries-extra}

\usepackage[
 % automake,% create glossaries automatically
 acronym,% create 'abbreviations' glossary
 index,% create 'index' glossary
 nostyles,%don't load predefined styles
 % postdot,% insert dot after descriptions
 % record,% using bib2gls
 stylemods={tree,bookindex},% load the 'tree' and 'bookindex' style packages
 style={tree},% set the default style to 'tree'
 toc% add glossaries to table of contents
]{glossaries-extra}


% ---------------------------------
% Preparing for use with Index
% ---------------------------------
\GlsXtrLoadResources[
 src={terms},% data in terms.bib
 label-prefix={idx.},% prefix for primary entry labels
 dual-prefix={},% prefix for dual entry labels
 type=index,% put primary entries in 'index' glossary
 combine-dual-locations={primary}% merge locations and assign to primary list
]

% provide commands that work like \gls etc for the @index entries
% (that don't have a dual counterpart)
\glsxtrnewglslike{idx.}{\idx}{\idxpl}{\Idx}{\Idxpl}

% ========================
% Figures
% ========================
% Subfigures via `\subfloat{...}`
\usepackage{subfig}

% Full page figures
\renewcommand{\topfraction}{.9}%{.85}
\renewcommand{\bottomfraction}{.8}%{.7}
\renewcommand{\textfraction}{.15}
\renewcommand{\floatpagefraction}{.66}
\renewcommand{\dbltopfraction}{.66}
\renewcommand{\dblfloatpagefraction}{.66}
\setcounter{topnumber}{9}
\setcounter{bottomnumber}{9}
\setcounter{totalnumber}{20}
\setcounter{dbltopnumber}{9}


% ========================
% Graphics
% ========================
\usepackage{graphbox,xspace,float,xcolor,color}

\usepackage{tikz, tikzscale, tikz-cd, pgf, pgfplots}  % drawing pictures
\usetikzlibrary{angles,quotes,shapes}

\pgfplotsset{compat=newest}
\usepgfplotslibrary{fillbetween}

% tikz styles
\tikzset{
  every overlay node/.style={
    draw=black,fill=white,rounded corners,anchor=north west,
  },
}
\def\tikzoverlay{%
   \tikz[baseline,overlay]\node[every overlay node]
}

% -----------------------------
% Tikz Flowchart:
% -----------------------------
% Define block styles
\tikzstyle{startstop} = [rectangle, rounded corners, minimum width=3cm, minimum height=1cm,text centered, draw=black, fill=red!30]
\tikzstyle{process} = [rectangle, minimum width=3cm, minimum height=1cm, text centered, draw=black, fill=orange!30]
\tikzstyle{decision} = [rectangle, minimum width=3cm, minimum height=1cm, text centered, draw=black, fill=green!30]
\tikzstyle{arrow} = [thick,->,>=stealth]

% ========================
% Symbols
% ========================
\DeclareMathSymbol{\mhyphen}{\mathord}{AMSa}{"39}

\usepackage{pifont}
\newcommand{\cmark}{\textcolor{green!60!black}{\ding{51}}}  % checkmark
\newcommand{\xmark}{\textcolor{red}{\ding{55}}}  % x mark
\newcommand{\oxmarksmall}{\textcolor{yellow!75!black}{\fontencoding{U}\fontfamily{futs}\selectfont\char 66\relax}}  % yellow warning triangle

\usepackage{contour}
\newcommand{\oxmark}{\protect\contourlength{.005pt}\protect\contour{yellow!75!black}{\oxmarksmall}} % Slightly bigger border

% --------------------------
% Note: \oxmark doesn't work in TeXLive, and therefore not in the updated ArXiv submission system
% --------------------------
% Alternative to oxmark
\newcommand\dangerthin{%
 \makebox[1.4em][c]{%
 \makebox[0pt][c]{\raisebox{-0.05em}{\textbf{\tiny\textcolor{yellow!75!black}!}}}%
 \makebox[0pt][c]{\raisebox{-0.2em}{\color{yellow!75!black}\Large$\mathbf{\bigtriangleup}$}}}}%
\newcommand\danger{\protect\contourlength{.13pt}\protect\contour{yellow!75!black}{\dangerthin}} % Slightly bigger border


% =====================
% Colors
% =====================
% Pastel primaries
\definecolor{amber}{rgb}{1.0, 0.75, 0.0}
\definecolor{cornflowerblue}{rgb}{.53, .80, .93}  % slightly different from online val, to match matplotlib
\definecolor{awesome}{rgb}{1.0, 0.13, 0.32}

\definecolor{darkpastelred}{rgb}{0.76, 0.23, 0.13}
\definecolor{ballblue}{rgb}{0.13, 0.67, 0.8}
\definecolor{azure}{rgb}{0.0, 0.5, 1.0}
\definecolor{darkspringgreen}{rgb}{0.09, 0.45, 0.27}
\definecolor{ashgrey}{rgb}{0.7, 0.75, 0.71}
\definecolor{aurometalsaurus}{rgb}{0.43, 0.5, 0.5}
\definecolor{babyblueeyes}{rgb}{0.63, 0.79, 0.95}
\definecolor{royalblue}{RGB}{99,149,236}  % slightly different from online val, to match matplotlib
\definecolor{dodgerblue}{rgb}{0.12, 0.56, 1.0}
\definecolor{forestgreen(traditional)}{rgb}{0.0, 0.27, 0.13}
\definecolor{forestgreen(web)}{rgb}{0.13, 0.55, 0.13}
\definecolor{darkgoldenrod}{rgb}{0.72, 0.53, 0.04}
\definecolor{goldenrod}{rgb}{0.85, 0.65, 0.13}
\definecolor{ochre}{rgb}{0.8, 0.47, 0.13}
\definecolor{plum}{RGB}{221,160,221}


% ========================
% Additional Paper Commands:
% ========================
\DeclareRobustCommand{\Sec}[1]{Sec.~\ref{sec:#1}}
\DeclareRobustCommand{\Secs}[2]{Secs.~\ref{sec:#1} and \ref{sec:#2}}
\DeclareRobustCommand{\Secss}[3]{Secs.~\ref{sec:#1}, \ref{sec:#2} and \ref{sec:#3}}
\DeclareRobustCommand{\App}[1]{App.~\ref{app:#1}}
\DeclareRobustCommand{\Apps}[2]{Apps.~\ref{app:#1} and~\ref{app:#2}}
\DeclareRobustCommand{\Appss}[3]{Apps.~\ref{app:#1},~\ref{app:#2},~and~\ref{app:#3}}
\DeclareRobustCommand{\Tab}[1]{Table~\ref{tab:#1}}
\DeclareRobustCommand{\Tabs}[2]{Tables~\ref{tab:#1} and \ref{tab:#2}}
\DeclareRobustCommand{\Fig}[1]{Fig.~\ref{fig:#1}}
\DeclareRobustCommand{\Figs}[2]{Figs.~\ref{fig:#1} and \ref{fig:#2}}
\DeclareRobustCommand{\Figss}[3]{Figs.~\ref{fig:#1}, \ref{fig:#2}, and \ref{fig:#3}}
\DeclareRobustCommand{\Eq}[1]{Eq.~(\ref{eqn:#1})}
\DeclareRobustCommand{\Eqs}[2]{Eqs.~(\ref{eqn:#1}) and (\ref{eqn:#2})}
\DeclareRobustCommand{\Eqss}[3]{Eqs.~(\ref{eqn:#1}), (\ref{eqn:#2}), and (\ref{eqn:#3})}
\DeclareRobustCommand{\Reff}[1]{Ref.~\cite{#1}}
\DeclareRobustCommand{\Reffs}[1]{Refs.~\cite{#1}}

\DeclareRobustCommand{\Thm}[1]{Theorem~\ref{#1}}

\newtheorem{definitiontext}{Definition}
\DeclareRobustCommand{\Def}[1]{Definition~\ref{def:#1}}


%=======================================
% Math Commands:
%=======================================
\usepackage{slashed}

%--------------------------------------
% Physics:
%--------------------------------------
\newcommand{\superfield}[1]{\ensuremath{\mathbf{#1}}}
\newcommand{\dfour}[1]{\ensuremath{\delta^{(4)}\le(#1\ri)}}

%--------------------------------------
% Math:
%--------------------------------------
% Basics
\newcommand{\p}{\partial}
\newcommand{\la}{\left\langle}
\newcommand{\ra}{\right\rangle}
\newcommand{\sgn}{{\rm sgn}}
\newcommand{\sinc}{{\rm \, sinc\,}}
\newcommand{\braaket}[3]{\left\langle #1 \right| #2 \left|#3 \right\rangle}
\def\nn{\nonumber\\}
\newcommand{\mpar}[1]{\marginpar{\small \it #1}}

% pmtrx (removed vowels): enclose argument in pmatrix, for easy row/col vecs and matrices
\newcommand{\pmtrx}[1]{
    \ensuremath{\begin{pmatrix}#1\end{pmatrix}}
}

\newcommand{\vstag}{{\bm u}}

% Blackboard bold
\renewcommand{\AA}{\ensuremath{\mathbb A}}
\newcommand{\CC}{\ensuremath{\mathbb C}}
\newcommand{\DD}{\ensuremath{\mathbb D}}
\newcommand{\EE}{\ensuremath{\mathbb E}}
\newcommand{\FF}{\ensuremath{\mathbb F}}
\newcommand{\HH}{\ensuremath{\mathbb H}}
\newcommand{\KK}{\ensuremath{\mathbb K}}
\newcommand{\NN}{\ensuremath{\mathbb N}}
\newcommand{\PP}{\ensuremath{\mathbb P}}
\newcommand{\QQ}{\ensuremath{\mathbb Q}}
\newcommand{\RR}{\ensuremath{\mathbb R}}
\newcommand{\ZZ}{\ensuremath{\mathbb Z}}
\newcommand{\Zt}{\ensuremath{\mathbb Z_2}}
\newcommand{\Zn}{\ensuremath{\mathbb Z_N}}

% Mathcal
\newcommand{\A}{\ensuremath{\mathcal A}}
\newcommand{\C}{\ensuremath{\mathcal C}}
\newcommand{\D}{\ensuremath{\mathcal D}}
\newcommand{\E}{\ensuremath{\mathcal E}}
\newcommand{\F}{\ensuremath{\mathcal F}}
%\newcommand{\H}{\ensuremath{\mathcal H}}
\newcommand{\K}{\ensuremath{\mathcal K}}
\newcommand{\Lag}{\ensuremath{\mathcal Lag}}
\newcommand{\N}{\ensuremath{\mathcal N}}
\renewcommand{\P}{\ensuremath{\mathcal P}}
\newcommand{\Q}{\ensuremath{\mathcal Q}}
\newcommand{\R}{\ensuremath{\mathcal R}}
\newcommand{\Z}{\ensuremath{\mathcal Z}}

% various math expressions that shouldn't be italicized
\DeclareMathOperator{\cis}{cis}
\DeclareMathOperator{\Var}{Var}
\DeclareMathOperator{\Cov}{Cov}
\DeclareMathOperator*{\lcm}{lcm}
\DeclareMathOperator{\ord}{ord}
\DeclareMathOperator{\Ker}{Ker}
\DeclareMathOperator{\Bin}{Bin}
\DeclareMathOperator{\res}{res}
\DeclareMathOperator{\rad}{rad}
\DeclareMathOperator{\Spec}{Spec}
\DeclareMathOperator{\Sing}{Sing}
\DeclareMathOperator{\ob}{ob}
\DeclareMathOperator{\Hom}{Hom}
\DeclareMathOperator{\chAb}{chAb}
\DeclareMathOperator{\Tor}{Tor}
\DeclareMathOperator{\Ext}{Ext}

% can't use DeclareMathOperator because already defined
\renewcommand{\Re}{\operatorname{Re}}
\renewcommand{\Im}{\operatorname{Im}}
\renewcommand{\ker}{\operatorname{ker}}
\renewcommand{\mod}[1]{\ \text{mod}\ #1}

% other convenient abbreviations
\newcommand{\eps}{\varepsilon}
\DeclareMathOperator{\indep}{\perp\!\!\!\perp}

% New definition of square root:
% Rename \sqrt as \oldsqrt
\let\oldsqrt\sqrt
% Define the new \sqrt in terms of the old one
\def\sqrt{\mathpalette\DHLhksqrt}
\def\DHLhksqrt#1#2{%
\setbox0=\hbox{$#1\oldsqrt{#2\,}$}\dimen0=\ht0
\advance\dimen0-0.2\ht0
\setbox2=\hbox{\vrule height\ht0 depth -\dimen0}%
{\box0\lower0.4pt\box2}}

% Misc. math definitions
\newcommand{\eqdelta}{\overset{\Delta}{=}}
\newcommand{\eqquestion}{\overset{?}{=}}
\newcommand{\eqexclamation}{\overset{?}{=}}
\newcommand{\reppedby}{\overset{\boldsymbol{\cdot}}{=}}
\newcommand{\reps}{\underset{\boldsymbol{\cdot}}{=}}
\newcommand{\paren}[1]{\left(#1\right)}
\newcommand{\pnorm}[1]{\left|\left|#1\right|\right|}
\newcommand{\floor}[1]{\left\lfloor#1\right\rfloor}
\newcommand{\mc}[1]{\ensuremath{\mathcal{#1}}}
\newcommand{\parsh}[3]{\left(\frac{\partial #1}{\partial #2}\right)_{#3}}
\newcommand{\opn}{\operatorname}
\newcommand{\qedblack}{
    \begin{flushright}
    \(\blacksquare\)
    \end{flushright}
}
% Sum-Integral
\DeclareMathOperator*{\SumInt}{%
\mathchoice%
  {\ooalign{$\displaystyle\sum$\cr\hidewidth$\displaystyle\int$\hidewidth\cr}}
  {\ooalign{\raisebox{.14\height}{\scalebox{.7}{$\textstyle\sum$}}\cr\hidewidth$\textstyle\int$\hidewidth\cr}}
  {\ooalign{\raisebox{.2\height}{\scalebox{.6}{$\scriptstyle\sum$}}\cr$\scriptstyle\int$\cr}}
  {\ooalign{\raisebox{.2\height}{\scalebox{.6}{$\scriptstyle\sum$}}\cr$\scriptstyle\int$\cr}}
}

\newcommand\half{{\ensuremath{\frac{1}{2}}}}
\def\le{\left}
\def\ri{\right}
\newcommand\da{{\dagger}}


%=======================================
% Theorems and Theorem Environments:
%=======================================
\newcommand{\vocab}[1]{\textcolor{blue!50!black}{\textbf{#1}}}

% colored boxes
\usepackage[most]{tcolorbox}
    \tcbuselibrary{theorems}
    \newtcolorbox{answerbox}{sharp corners=all, colframe=black, colback=black!5!white, boxrule=1.5pt, width = 1\textwidth, valign=center}
    \newenvironment{answer}{\begin{center}\begin{answerbox}}{\end{answerbox}\end{center}}

\definecolor{defcolor}{rgb}{0.8, 0.0, 0.0} % Boston University Red
\definecolor{thmcolor}{rgb}{0.0, 0.28, 0.67} % Cobalt
\definecolor{propcolor}{rgb}{0.69, 0.4, 0.0} % Ginger
\definecolor{qcolor}{rgb}{0.6, 0.2, 0.8} % Dark Orchid


% Tools for theorem boxes:
\usepackage{thmtools}
\usepackage[framemethod = TikZ]{mdframed}
\usepackage{silence} % for suppressing warnings
\WarningFilter{mdframed}{You got a bad break}

% Counter for theorem boxes:
\newcounter{thmcounter}[section]
\newcounter{probcounter}[section]

%%%%%%%%%%%%%%%%%%%%%

\mdfsetup{
	linewidth = 0.3mm,
	innertopmargin = 2mm,
	innerbottommargin = 3.5mm,
	innerleftmargin = 3mm,
	innerrightmargin = 3mm
} % adjusts boundaries of boxes

\newcommand{\thmboxstyle}[4]{
	\mdfdefinestyle{#2}{
		linecolor = #3,
		backgroundcolor = #4,
		nobreak = true
	}
	\declaretheoremstyle[
		headfont = \sffamily\bfseries\color{#3},
		mdframed = {style = #2},
		headpunct = {\\[0.4pt]},
		postheadspace = {0pt},
	]{#1}
}


%--------------------------------------
% tcb Theorem Styles:
%--------------------------------------
% Definition
\newtcbtheorem[auto counter, number within=section]
{definitionbox}{Definition}{
    enhanced,
    breakable,
    colback=defcolor!5!white,
    colframe=defcolor!75!black,
    colbacktitle=defcolor!17!white,
    coltitle=defcolor!70!black,
    fonttitle=\bfseries,
    lower separated=false,
    boxed title style={colframe=defcolor!75!black},
    attach boxed title to top left={xshift=0.5cm,yshift=-2mm},
    title={Definition \thetcbcounter: #1}
}{def}

% Theorem
\newtcbtheorem[number within=section, use counter from=definitionbox]
{theorembox}{Theorem}{
    enhanced,
    breakable,
    colback=thmcolor!5!white,
    colframe=thmcolor!75!black,
    colbacktitle=thmcolor!17!white,
    coltitle=thmcolor!70!black,
    fonttitle=\bfseries,
    lower separated=false,
    boxed title style={colframe=thmcolor!75!black},
    attach boxed title to top left={xshift=0.5cm,yshift=-2mm},
    title={Theorem \thetcbcounter: #1}
}{thm}

% Proposition
\newtcbtheorem[number within=section, use counter from=definitionbox]
{proposition}{Proposition}{
    enhanced,
    breakable,
    colback=propcolor!5!white,
    colframe=propcolor!75!black,
    colbacktitle=propcolor!17!white,
    coltitle=propcolor!70!black,
    fonttitle=\bfseries,
    lower separated=false,
    boxed title style={colframe=propcolor!75!black},
    attach boxed title to top left={xshift=0.5cm,yshift=-2mm},
    title={Proposition \thetcbcounter: #1}
}{prop}

% Question
\newtcbtheorem[number within=section, use counter from=definitionbox]
{question}{Question}{
    enhanced,
    breakable,
    colback=qcolor!5!white,
    colframe=qcolor!75!black,
    colbacktitle=qcolor!17!white,
    coltitle=qcolor!70!black,
    fonttitle=\bfseries,
    lower separated=false,
    boxed title style={colframe=qcolor!75!black},
    attach boxed title to top left={xshift=0.5cm,yshift=-2mm},
    title={Question \thetcbcounter: #1}
}{question}

% Remark
\newtcbtheorem[number within=section, use counter from=definitionbox]
{remarkbox}{Remark}{
    enhanced,
    breakable,
    fonttitle=\bfseries,
    lower separated=false,
    standard jigsaw,
    opacityback=0,
    opacitybacktitle=0,
    opacityframe=0,
    coltitle=ao!90!black,
    title={Remark \thetcbcounter: #1}
}{rem}

\newcommand{\remark}[2]{
\begin{remarkbox}{}{#1}
    \textit{#2}
\end{remarkbox}
}%

% Goal
\newtcolorbox{goalbox}[2][]{
    colback=goalcolor!5!white,
    colframe=goalcolor!75!black,
    colbacktitle=goalcolor!10!white,
    coltitle=goalcolor!70!black,
    title={#2},fonttitle=\bfseries,#1}


% ========================
% Misc.
% ========================
\usepackage{hhline}

\pdfoutput=1
% Prevent footnotes from breaking into two pages
\interfootnotelinepenalty=10000
