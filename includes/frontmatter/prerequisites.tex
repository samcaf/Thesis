\section*{Prerequisites}
\markboth{Prerequisites}{}

\epigraph{``You earn your right to speculate depending on how much work you do.''}{Barton Zweibach}

In order to understand the technical details of the material in this thesis, I expect that the following prerequisites are necessary (if you can understand them without these prerequisites, great work!):
\begin{itemize}
    \item
        A course in quantum field theory, or at least an understanding of quantum mechanics and the basics of particle physics.

    \item

\end{itemize}

In addition, I expect that you should probably be able to understand the following pieces of physics:
\begin{itemize}
    \item
        The strong coupling \(\alpha_s(\mu)\) obeys the \vocab{renormalization group equation}
        \begin{equation}
        \mu \frac{d}{d\mu} \alpha_s(\mu) = \beta(\alpha_s(\mu))
        =
        - \sum_{n=0}^\infty \beta_n
        {\left(\frac{\alpha_s(\mu)}{4\pi}\right)}^{n+2},
        \end{equation}
        \sam{check} with the first \vocab{beta function coefficient}
        \begin{equation}
            \beta_0
            =
            \frac{1}{4\pi}\le(
                \frac{11}{3} C_A - \frac{4}{3} T_F n_f
            \ri)
            ,
        \end{equation}
        taking an important role at one-loop and in the studies in this thesis.

    \item
        The running of the strong coupling leads to a special scale by \vocab{dimensional transmutation}:
        %
        given the coupling at some scale \(\mu\), the form of the beta function indicates that the coupling becomes infinite at a scale we call \(\LambdaQCD\).
        %
        Ignoring the effects of the beta-function coefficients other than \(\beta_0\), we have
        \begin{equation}
            \LambdaQCD = \mu \exp\left(-\frac{1}{2\beta_0 \alpha_s(\mu)}\right).
        \end{equation}
        \sam{This was just written by copilot -- double check in terms of these definitions.}
\end{itemize}
