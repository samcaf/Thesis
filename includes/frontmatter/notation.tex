% =====================================
\section*{Notation and Terminology}
% =====================================
\markboth{Notation and Terminology}{}

% =====================================
\begin{sambox}{Resources}{}
    % ---------------------------------
    \begin{itemize}
        \item
            Larkoski Chapter 9:
            %
            ee to hadrons, jets, splitting functions, pheno problems

        \item
            Foundations Chapter 8:
            %
            Factorization, Mellin, DGLAP, sum rules

        \item
            Substructure Chapter 1:
            %
            Substructure motivation
    \end{itemize}
    % ---------------------------------
\end{sambox}
% =====================================

\begin{sambox}{Some TODOs for Sam}{}
    \begin{itemize}
        \item
            Is there a proof that pdfs and fragmentation functions are universal?
            %
            Should put this in somewhere...

        \item
            \sam{Does fragmentation belong in a place where pdfs can fit in..?
                %
                Seems like factorization is the actual important piece. But my thesis is not about factorization!
                %
                I need to find a tasteful way to include it.
            }

        \item
            \sam{Need to fix header on problem pages, which simply use the most recent appendix instead.}
    \end{itemize}
\end{sambox}

Here is some common, and relatively important, notation used throughout this thesis.
%
See \Sec{morenotation} for a more comprehensive list.

\begin{itemize}
    \item
        A \vocab{calcule} is a small calculation, often a simple example or a part of a bigger calculation.

    \item
        \(\eqdelta\) denotes equality by definition;

    \item
        For notational simplicity, I will set the physical constants \(\hbar = c = 1\) in all printed equations;

    \item
        \sam{Metric signature}

    \item
        The Mellin transform of a function \(f(x)\) on the interval $(0,\infty)$ is defined as
        \begin{align}
            \mathcal{M}[f](s)
            \eqdelta
            \int_0^\infty \,\frac{\dd x}{x}\, x^{s-1} \, f(x)
            \eqdelta
            \hat f(s)
            .
        \end{align}

        For all of the applications in this work, other than some derivations which are relegated to the Problems \sam{of Chapter...}, the functions in whose Mellin transforms we are interested will have support only on the interval \((0,1)\).
        %
        Therefore, unless stated otherwise, the Mellin transform appearing in this work will appear in the form
        \begin{align}
            \hat f(s)
            \eqdelta
            \int_0^1 \,\frac{\dd x}{x}\, x^{s-1} \, f(x)
            .
        \end{align}

        \sam{Put into the text where relevant}

        \sam{Mention here the part of the text where this appears and the problems in which it appears}

    \item
        \sam{LO, NLO, LL, NLL, etc.}

    \item
        The \textit{energy fraction} carried by a particular particle in a scattering event or jet, relative to the total energy of the event or jet, will play an important role in this thesis.
        %
        We will use the symbol \(z_i\) to denote the energy fraction particle \(i\) within a jet or event:
        \begin{align}
            z_i
            \eqdelta
            \frac{E_i}{E_\text{tot}}
            .
        \end{align}

    \item
        A useful definition for clarity is that of the value of \(\alpha_s\) divided by \(4\pi\), since this quantity appears in loop computations (which often have a \(4\pi\) suppression):
        \begin{subequations}
        \begin{align}
            a_s = \frac{\alpha_s}{4\pi}
        \end{align}
        \end{subequations}
\end{itemize}
