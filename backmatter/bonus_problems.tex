\section*{Mellin Transforms}

\begin{sambox}{Some resources}{}
% https://users.dimi.uniud.it/~giacomo.dellariccia/Glossary/transforms/Oosthuisen2011.pdf

    \href{https://users.dimi.uniud.it/~giacomo.dellariccia/Glossary/transforms/Bertrand\%20J.Bertrand\%20P.Ovarle.2000.pdf}{Bertrand, Ovarlez, 2000}

    \href{https://en.wikipedia.org/wiki/Integral_transform}{Wikipedia: Integral Transform}

    \href{https://en.wikipedia.org/wiki/Fourier_transform_on_finite_groups}{Fourier transform on finite groups}

    \href{https://mathoverflow.net/questions/79868/what-does-mellin-inversion-really-mean}{MathOverflow: What does Mellin inversion really mean?}

    \href{https://people.cs.uchicago.edu/~laci/reu02/fourier.pdf}{The Fourier Transform and Equations over Finite Abelian Groups}

    \href{https://golem.ph.utexas.edu/category/2010/11/integral_transforms_and_pullpu.html}{Integral Transforms and the Pull-Push Perspective, I}

    \href{https://en.wikipedia.org/wiki/Inverse_scattering_transform}{Inverse scattering transform (Nonlinear generalizations)}

    \href{https://en.wikipedia.org/wiki/Schwartz_kernel_theorem}{Schwartz kernel theorem}

    \href{https://math.stackexchange.com/questions/501899/importance-of-schwartz-kernel-theorem}{Math StackExchange: Importance of Schwartz kernel theorem}

    \href{https://math.stackexchange.com/questions/2623515/schwartz-kernel-theorem-and-dual-topologies}{Math StackExchange: Schwartz kernel theorem and dual topologies}

    \vspace{.5em}

    \textbf{Applications?}

    \href{https://watermark.silverchair.com/748_1_online.pdf?token=AQECAHi208BE49Ooan9kkhW_Ercy7Dm3ZL_9Cf3qfKAc485ysgAABWswggVnBgkqhkiG9w0BBwagggVYMIIFVAIBADCCBU0GCSqGSIb3DQEHATAeBglghkgBZQMEAS4wEQQMp8SxgPn6c7OBLlGhAgEQgIIFHmy00C8eTOUHPbWo8YSihdv0C7NlSfT03xtgMgSAn8fZC5ht3MMECj9-CC6UPP0eJmNWuIgvSj0GaqPDlIC5_WCruiOyqTClvbnr3yPMmV50MYWekQ-cr2ZXoQPJHc2bnnB1DEPxsEaO_PvF5u_2pE4jjqf0CG4JGhHVLPyEpciWGpE1Kh5AlOVxGqy7d7OoL73SJ5obxDW0VztTBhEHp8Z2UDNqGx9beHfyjs62oaMZrdIxAW-zO2UeC53xm1YK-o42GA4WVxeNJndfcHgO-Div8Aa0wSAZc928-vtnhb2sEep9kAGB0lf8a_VCWdY5oCbvS5R1Qls0kuNpdIwLzRo5jM7XBiX1G0hOGzjSPYfvlhlh-xSOP9l17TJ8r1uGyBK4yukjK1SBBkMx67uE3oAzHrMNNx1L1Y6KzYdHeBZQqHB_uB23XbICWBcjfG1Q0U4Esk0MLqMwCL7FvfR25qlfDyDBgypeKopm80aC16fyFdgxtYs7M_bvit4-UK-0dhJGek9ROuQWueXdiThX7MnlWoZ2680UmF49sl5SLlqccH4AGFEEa3LHYVkeUlG6cJhVBmLXBCFpLJm2hcV7spbzEp8RMihe1yAo75YlCnVPNMfSHIggSEatpaYnSFYb5XymdnD1oWj4mQu7DVjnyNi3nzpxGq9Wm0yRXleWhNUMjg-BYQHUe6m2DZ2-zp3sHrvCEywoTOob054ONhNlXbQfwzFyTZrZQ5UtI9UIfb-tRm9FknCpMu9r6FtBT--OYfbNU7pf7d2Vf2YSKu3UmwQ6yyyc_h6qXB4dHdPMnyGKN0rIvw0rJ_KK_qkTwNeEUkwuIltITMCW9UwVchCXjFqpCOf32Ag5N8383O1jBaOEeswn4wz9DH0b1qQW5304289zPqddB6BmsCqP-y7Z-iq-0gjif7e-gjuyBxqIm9OuW90lZqdzkZS8uElGHlYHRseRetE09YqoLzEwBzAv1Xa2vqx0L8NkXOUQdYQGlrR1bk_Bmim7WsPpGwcirDUgb_NMnV5exHa8xPcgpEH5NGiOTbIbaxiCa8j9-1ULa-ds_-vLGAJhT3PA32wfkI2zwtlfCg-fMuvzDFOfFOy0eMkXS3CDd8cRNeLbdXOqc43wSs9Vj3lVWzP-PDYI_zX44fXkpFKAVdv1GAYRnZrCg36rR-Af8yNr7X_9MnbaJH2t05hKPTQsRERXRTtOHFe5kNaliQevn2LmoJEI4D7Kp4htw_kRoTDVsmTe4hRbPuHL0xHMH8si0OH6_l1ji_WM5IfLR6tImH5-7hcF_EizivLaxAAcEXfQ0JEvoR1-kVzqErfrV2CxDhkOMYb3QkJHLxVOCIMfEmTngkzD2Xtm1QDD-i7yd6Oyba1UH4ODMIxm5eFVLBT6s3FV3D5qYDCmwhWftns93iz135CcrvWkuWaTGaGK3NyAYwIe_Z5WuG2MeTwKD6ILw8N6iJ0WB4icc5mf9OBeb3FCXGgW6W_H80nFxTFEP4Dk-cRWGLPgMyIDxpNVgupWdj7qT-0bCGGRzfKxPEU05bx3Io_uTzWEwCeapvu_4eWK9AddxzOLGtkvkr0O3m34_8QvD-ww-VilIdoXe189dBVRb1g3knDQHbt59Yqq1LKZwvcH9Zr9JE49qN1xyFJZnbsIGaYzHPOrw6pgVfoatpOIzTtivOfdGw6SfeKc9rwUReNp7az0xPBPe7TrH-nNuycvDM-a5klPwcG8dc2rDcb9gaSyN3DP}%
    {The power spectrum of the Mellin transformation with applications to scaling of physical quantities}

    \href{https://phsites.technion.ac.il/eric/wp-content/uploads/sites/6/2019/02/Ariane_Soret_MSc_Thesis.pdf}{Quantum dynamics for a fractal spectrum (thesis)}

    \href{https://www.ncbi.nlm.nih.gov/pmc/articles/PMC3222220/}{A Biologically Plausible Transform for Visual Recognition that is Invariant to Translation, Scale, and Rotation}
\end{sambox}




\makebonusprob{What is the Mellin Transform?}{mellinintro}{
    \sam{Problem regarding Mellin transforms}

    \sam{Check signs and prove}

    \textbf{Warmup by Analogy: Fourier Transform}

    In the Fourier transform, a central role is played by the \emph{translations}
    \begin{align}
        x \mapsto x + a
        .
    \end{align}
    The Fourier transform of a function \(f(x)\) divides the function \(f(x)\) into ``plane-wave'' components which are eigenstates of translation.

    The Fourier transform of a function \(f(x)\) is defined as
    \begin{align}
        \mc F[f(x)](k)
        \eqdelta
        \tilde f(k)
        &\eqdelta
        \int_{-\infty}^{\infty} \dd x \, e^{-ikx} f(x)
        .
    \end{align}

    \begin{enumerate}[label=\roman*)]
        \item
            What is the differential operator \(\mc T\) which implements an infinitesimal translation in \(x\)?
            %
            In particular, we are looking for \(\mc T\) such that:
            \begin{align}
                f(x) + \delta x \, \mc T f(x) = f(x+\delta x)
                .
            \end{align}

        \item
            Argue that the Fourier transform of \(f(x)\) can be divided into three conceptually distinct pieces:
            \begin{itemize}
                \item
                    Integration against a translation-invariant measure;

                \item
                    An eigenstate of the translation operator;

                \item
                    The function \(f(x)\) itself.
            \end{itemize}

        \item
            Using the invariance of the integration measure under translations, show that the Fourier tranform of \(f(x)\) after translation by \(a\) is simply related to the Fourier transform of \(f(x)\):
            \begin{align}
                \mc F[f(x+a)](k) = e^{ika} \tilde f(k)
                .
            \end{align}

        \item
            Show that the Fourier transform of \(\mc T f(x)\) is simply related to the Fourier transform of \(f(x)\).
    \end{enumerate}

    \vspace{.5cm}

    \textbf{Finally: Mellin Transform}

    The Mellin transform is similar to the Fourier transform, except that it is now \emph{dilations} which play a central role:
    \begin{align}
        x \mapsto \lambda x
        .
    \end{align}
    The Mellin transform of a function \(f(x)\) divides the function \(f(x)\) into analogous components which are eigenstates of dilation.

    The Mellin transform of a function \(f(x)\) was defined in \Def{mellin}, and takes the form
    \begin{align}
        \mc M[f(x)](s)
        \eqdelta
        \hat f(s)
        &\eqdelta
        \int_0^\infty \dd x \, x^{s-1} f(x)
        .
    \end{align}

    \begin{enumerate}[label=\roman*)]
        \item
            What is the differential operator \(\mc D\) which implements an infinitesimal \vocab{dilation} in \(x\)?
            %
            In particular, we are looking for \(\mc D\) such that:
            \begin{align}
                f(x) + \delta \lambda \, \mc D f(x)
                =
                f\le(\le(1 + \delta \lambda\ri)x\ri)
                .
            \end{align}

        \item
            Argue that the Mellin transform of \(f(x)\) can be divided into three conceptually distinct pieces:
            \begin{itemize}
                \item
                    Integration against a dilation-invariant measure;

                \item
                    An eigenstate of the dilation operator;

                \item
                    The function \(f(x)\) itself.
            \end{itemize}

        \item
            Using the invariance of the integration measure under dilations, show that the Mellin tranform of \(f(x)\) after dilation by \(\lambda\) is simply related to the Mellin transform of \(f(x)\):
            \begin{align}
                \mc M[f(\lambda x)](s) = \lambda^{-s} \hat f(s)
                .
            \end{align}
    \end{enumerate}
}



\makebonusprob{Mellin Convolution Theorem}{mellinconvolution}{
    As you will now show, the Mellin transform of two functions can be naturally associated with a convolution of the form
    \begin{align}
        (f \mellinconvolution g)(x)
        =
        \int_0^\infty \frac{\dd y}{y}\, f(y) g(x/y)
        .
    \end{align}

    Prove that the Mellin transform of a convolution of two functions is the product of the Mellin transforms of the individual functions:
    \begin{align}
        \mathcal{M}[f \mellinconvolution g](s)
        =
        \mathcal{M}[f](s) \, \mathcal{M}[g](s)
        .
    \end{align}
    This result will be important in understanding how \sam{certain quantities interact with the DGLAP evolution equations}, and \sam{we can create some physical intuition for this result in the following problem.}

    \sam{State the theorem when applied on functions only with support from 0 to 1}

    \sam{Still want to figure out what exactly is happening here (intuition) -- in diagonalization of dilatation and so on}

    \sam{Could be worthwhile to again compare to Fourier}
}

\makebonusprob{Mellin Convolution Theorem II}{mellinconvolution2}{
    There is another type of convolution that can be expressed simply with Mellin transforms.
    %
    Let us consider the new convolution
    \begin{align}
        (f \circ_{\mc M} g)(x)
        =
        \int_0^\infty\,\dd\xi f(x\,\xi) g(\xi)
        .
    \end{align}
    Show that the Mellin transform of this convolution is
    \begin{align}
        \mathcal{M}[f \circ_{\mc M} g](s)
        \,
        =
        \,
        \mathcal{M}[f](s) \, \mathcal{M}[g](1-s)
        .
    \end{align}

    \sam{Where useful? Ever in jets?}

    \sam{State the theorem when applied on functions only with support from 0 to 1}

    \sam{Still want to figure out what exactly is happening here (intuition) -- in diagonalization of dilatation and so on}

    \sam{Could be worthwhile to again compare to Fourier}
}

\sam{Could be worthwhile to start with a simple power law to show scaling}

\makebonusprob{Abstract Examples of Mellin Transforms}{abstractmellin}{
    \sam{introduce \(f(x)\), \(b\)}
    \begin{enumerate}[label=\roman*)]
        \item
            \(x^b f(x)\)

        \item
            \(\frac{\dd}{\dd x} f(x)\)

        \item
            \(\int_0^x f(t) \dd t\)

        \item
            \(f(x^a)\)?
    \end{enumerate}

    \sam{Give intuition... scaling, etc.}
}


\makebonusprob{More Concrete Examples of Mellin Transforms}{concretemellin}{
    Perform the Mellin transforms of the following functions:
    \begin{enumerate}[label=\roman*)]
        \item
            \(x^n\) for \(x \in [0,1]\)

        \item
            \(\frac{1}{1 + x}\)

        \item
            \(e^{-x}\)

        \item
            \(e^{-a x}\) for \(\Re(a) > 0\), and thus \(\sin(k x)\) and \(\cos(k x)\)

        \item
            \(e^{-x^2}\)

        \item
            \(\frac{1}{1 + e^{-x}}\)
     \end{enumerate}

     \sam{Intuition -- compare some of the different examples...}

}


\makebonusprob{Ramanujan's Master Theorem}{ramanujanmastertheorem}{
    \sam{Problem regarding Ramanujan's master theorem}
}

\makebonusprob{Mellin Inversion Theorem}{mellininversion}{
    \sam{Problem regarding Mellin inversion theorem}
    \begin{align}
        f(x)
        =
        \frac{1}{2 \pi\, i}
        \int_{c - i \infty}{c + i\infty}
        \dd s \, x^{-s} \hat f(s)
    \end{align}
    \sam{Note that it's like a continuous analog of a power series}

    \sam{Hint Inverse Laplace and prove}
}

\makebonusprob{Mellin Transform of a Power Law}{mellinpowerlaw}{
    \begin{enumerate}[label=\roman*)]
        \item
            Fixed scaling -- inverse of delta function
    \end{enumerate}

    \sam{Give intuition... scaling, etc.}
}

\makebonusprob{Parseval's Theorem}{mellinparseval}{
    \sam{Problem regarding Parseval's theorem}

    \begin{align}
        \mc M [f(x) g(x)](s)
        =
        \frac{1}{2\pi\,i}
        \int_{c-i\infty}^{c+\infty}
        \dd p \, \hat f(p) \hat g(s-p)
    \end{align}
}

\makebonusprob{Perron's Formula}{perronformula}{
    \sam{Problem regarding Perron's formula}
}

\sam{Perhaps for later -- show that Mellin transform of the definition of a pdf is a local operator}

\sam{Ian mentions the paper 1011.1485}

\sam{I also found 1107.1499}

\textbf{Fractals:}

\sam{Borrowing the following from the thesis resource in the comments above}

\sam{Section 2.4 -- scaling properties}

\sam{Damn I like this thesis, super cool}
