\section*{Preface}
% \markboth{Preface}{Preface}

\epigraph{Question everything at least once.}{Mansour Alipour-fard}

\epigraph{You earn your right to speculate depending on how much work you do.}{Barton Zwiebach}


The nominal goal of this thesis is to present my research from my time in graduate school.

Another goal I have is that, in a couple of (albeit concerted) weeks, you can master what took my six years to learn (to the extent that I have mastered it)!
%
First, and most importantly, please remember that problems are a \textit{shortcut} to understanding, not ``something extra''.
%
At least, good problems are.
%
I've tried to select and invent some good problems for you%
\footnote{
    And if you have any comments or qualms regarding any of them, you should feel free to email me.
}
%
that I hope are representative of
\begin{itemize}
    \item
        the story or ``flavor'' of the material;

    \item
        the physical concepts, and the intuition I have for the material;

    \item
        the mathematics required to do some realistic (albeit simple) computations.
\end{itemize}
%
From what I've seen and read, active learning is the quickest and only way.
%
Don't forget!

Before we begin the thesis, I will add some clarifying comments regarding both goals.
%
First, my main contributions presented in this thesis are
\begin{itemize}
    \item
        The introduction of the \PIRANHA{} paradigm for continuous grooming, with Eric Metodiev, Patrick Komiske, and Jesse Thaler.
        %
        This research was presented in \ul{Pileup and Infrared Radiation Annihilation (PIRANHA): A Paradigm for Continuous Jet Grooming} \cite{Alipour-Fard:2023prj}.
        %
        For \PIRANHA{}, see especially \Secss{piranha}{sd_discont}{grooming-pheno};

    \item
        The development of new, efficient, and visually intuitive higher-point energy correlators -- both \emph{projected} (PENCs) and \emph{resolved} (RENCs) with Ankita Budhraja, Wouter Waalejwin, and Jesse Thaler.
        %
        This research was presented in \ul{New Angles on Energy Correlators} \cite{Alipour-fard:2024szj}.
        %
        For PENCs and RENCs, see \Sec{new-angles};

    \item
        The initiation of the study of general energy-weighted correlations, called Energy Weighted Observable Correlations (EWOCs), that are not restricted to angular correlations (e.g. as in the case of PENCs and RENCs).
        %
        This research was presented in \ul{Energy Correlators Beyond Angles} \cite{Alipour-fard:2025dvp}.
        %
        For EWOCs, see especially \Sec{ewocs}.
\end{itemize}

Second, since I attempt to provide especially clear and step-by-step derivations which highlight at least my own assumptions, several of the derivations presented in this thesis, even of fundamental concepts, are original.
%
Though their conclusions are backed by history, I invoke the uncertainty principle, and advise you to watch closely for their \emph{correctness}.
%
I hope to have given enough detail to make this possible, and I encourage you to email me if I have not.
%
The framework I derive has definite limitations, including (but not limited to):
\begin{itemize}
    \item
        There is no formal discussion of the ``accuracy'' of our computations -- we simply take the soft- and collinear-limits as often as we must to make progress;

    \item
        There is no systematic regularization of infrared divergences;
\end{itemize}
I would be delighted to hear if you know of any places where my discussion is lacking.

Finally, in addition to the several texts that appear in my citations below, another under-celebrated text on \glslink{qcd}{quantum chromodynamics} is \underline{Applications of Perturbative QCD}, by Richard D. Field \cite{Field:1989uq}, which pays incredible attention to detail in the presentation of calculations.
%
Of course, it would be a shame to access this book quickly, easily, and for free using \url{libgen.is}, so I will mention that it is also available (at time of writing) at \href{https://www.desy.de/~jung/qcd_and_mc_2009-2010/R.Field-Applications-of-pQCD.pdf}{\texttt{this desy.de link}}.

\begin{center}
Good luck.
\end{center}
