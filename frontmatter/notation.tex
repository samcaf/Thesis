% =====================================
\section*{Notation and Terminology}
% =====================================
% \markboth{Notation and Terminology}{Notation and Terminology}


\epigraph{If you think this Universe is bad, you should see some of the others.}{Philip K. Dick}

Here is some common notation used throughout this thesis.
\begin{itemize}
    \item
        \(A := B\) means ``\(A\) is defined to mean \(B\)'', as does \(B =: A\);

    \item
        For simplicity, I set the physical constants \(\hbar = c = 1\) in all printed equations;

    \item
        We make the unfortunate choice to use the ``mostly-minus'' metric, \(g^{\mu\nu} = \text{diag}\le(+,\,-,\,-,\,-\ri)\), such that \(\acomm{\gamma^\mu}{\gamma^\nu} = 2 g^{\mu\nu}\), and with the saving grace that we can write \(p^2 = m^2\).

    \item
        We will use the symbol \(z_i\) to denote the energy fraction particle \(i\) relative to a ``parent'' of any type, such as a pair of particles, a jet, or an entire event (which of these should be clear based on context):
        \begin{align}
            z_i
            \eqdelta
            \frac{E_i}{E_\text{tot}}
            \,.
        \end{align}


    \item
        We will also use the following standard definitions:%
        \footnote{
            We are leaving out the term \(-\epsilon \alpha_s\) that appears in \(d=4-2\epsilon\) dimensions when defining \(\alpha_s\) to be dimensionless.
            %
            The extra factor of two in the beta function vanishes if one defines the beta function instead to be the derivative of \(\alpha_s\) relative to \(\ln\mu^2\), or if it is written in terms of \(g_s\):
            %
            \(\dd g_s / \dd\ln_\mu = - g_s \sum_n \beta_n \le(\alpha_s/4\pi\ri)^{n+1}\).
        }
        \begin{subequations}
        \label{eq:qcd_coupling}
        \begin{align}
            \label{eq:as_defn}
            \alpha_s = \frac{g^2_s}{4\pi}
            \,,&
            \qquad\quad
            \as = \frac{\alpha_s}{4\pi}
            \,,
            \\
            \label{eq:beta_fn_defn}
            \beta(\alpha_s)
            :=
            \frac{\dd}{\dd\log\mu}\alpha_s&(\mu)
            =
            -2 \, \alpha_s
            \sum_{n=0}^\infty
            \beta_n \le(\frac{\alpha_s}{4\pi}\ri)^{n+1}
            \,.
        \end{align}
        At one loop (i.e. ignoring \(\beta_m\) for \(m \geq 1\)),
        \begin{align}
            \label{eq:alphas_1loop}
            \alpha_s^\text{(1-loop)}(\mu)
            &=
            \frac{\alpha_s(Q)}{1 + \alpha_s(Q) \beta_0 \log(\mu/Q)/2\pi}
            \,.
        \end{align}
        where
        \begin{align}
            \beta_0 = \frac{11}{3}\,C_A - \frac{4}{3}\,T_f N_f
        \end{align}
        \end{subequations}
        is the leading coefficient of the QCD beta function, \(N_f\) denotes the number of effectively massless quark flavors (\(N_f = 5\) in this thesis), \(Q\) indicates a reference scale (such as the energy scale of a hard process), and
        \begin{align}
        \label{eq:group_theory}
            C_F = \frac{N_c^2 - 1}{2N_c}
            \xrightarrow[]{\text{QCD}}
            \frac{4}{3}
            \,,
            \qquad
            C_A = N_c
            \xrightarrow[]{\text{QCD}}
            3
            \,,
            \qquad
            T_F = \frac{1}{2}
            \,,
        \end{align}
        denote the quadratic Casimir for the fundamental representation of \(SU(N_c)\), the quadratic Casimir for the adjoint representation, and the Dynkin index of the fundamental representation, respectively.
        %
        The expressions of \Eq{group_theory} are often referred to collectively as \textit{group theory factors}.
        %
        For QCD, \(N_c = 3\), \(C_F = 4/3\), and \(C_A = 3\), giving \(\beta_0 = 11 - 2 N_f / 3\,.\)


    \item
        The \glslink{splittingfn}{Dokshitzer-Gribov-Lipatov-Altarelli-Parisi (DGLAP) splitting functions}, computed in \Sec{splitting-functions} and related to the \glslink{dglapeqn}{DGLAP equation} in \Sec{p2p-fragmentation}, are often written in the \emph{exclusive} form \(P_{i\,j \leftarrow \ell}(z)\), with a probabilistic interpretation related to the probability of a parton of type \(\ell\) to split into partons of type \(i\), carrying an energy fraction \(z = z_i = E_i / E_\ell\), and type \(j\), carrying an energy fraction of \(z_j = 1-z\).
        %
        The \emph{inclusive} (meaning ``including a sum over some final state information'') DGLAP splitting functions are defined by
        \begin{align}
            \sum_{k'} P_{i\,k' \leftarrow \ell}(z) =: p_{i\leftarrow \ell}(z)
            \,.
        \end{align}


    \item
        The Mellin transform of a function \(f(x)\) with support on the interval \((0,1)\) (as in all concrete applications in this work) is
        \begin{align}
            \mathcal{M}[f](s)
            \eqdelta
            \int_0^1 \,\frac{\dd x}{x}\, x^{s-1} \, f(x)
            =
            \hat f(s)
            \,.
        \end{align}

\end{itemize}


Furthermore, we will assume
\begin{itemize}
    \item
    That all partons are massless other than the top quark (whose effects we will mostly ignore);

    \item
    That we may ignore electroweak effects (i.e. perform calculations without the \(W\) and \(Z\) bosons).
\end{itemize}
