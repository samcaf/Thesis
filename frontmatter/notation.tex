% =====================================
\section*{Notation and Terminology}
% =====================================
% \markboth{Notation and Terminology}{Notation and Terminology}


\epigraph{If you think this Universe is bad, you should see some of the others.}{Philip K. Dick}


% =====================================
\begin{sambox}{Resources}{}
    % ---------------------------------
    \begin{itemize}
        \item
            Larkoski Chapter 9:
            %
            ee to hadrons, jets, splitting functions, pheno problems

        \item
            Foundations Chapter 8:
            %
            Factorization, Mellin, DGLAP, sum rules

        \item
            Substructure Chapter 1:
            %
            Substructure motivation
    \end{itemize}
    % ---------------------------------
\end{sambox}
% =====================================

\begin{sambox}{Some TODOs for Sam}{}
    \begin{itemize}
        \item
            Need to fix header on problem pages, which simply use the most recent appendix instead.

        \item
            List of problems

        \item
            Signature page

        \item
            Add epsilon UV/IR calculations involving EEC

        \item
            Unify notation of epigraphs, probably want names then titles
    \end{itemize}
\end{sambox}

Here is some common notation used throughout this thesis.
%
See \Sec{morenotation} for a more comprehensive list.

\begin{itemize}
    \item
        A \vocab{calcule} is a smaller analogue of a calculation (such as a simple example or a part of a bigger calculation)

    \item
        \(A := B\) means ``\(A\) is defined to mean \(B\)'', as does \(B =: A\);

    \item
        For notational simplicity, I will set the physical constants \(\hbar = c = 1\) in all printed equations;

    \item
        \sam{Metric signature}

    \item
        The Mellin transform of a function \(f(x)\) on the interval $(0,\infty)$ is defined as
        \begin{align}
            \mathcal{M}[f](s)
            \eqdelta
            \int_0^\infty \,\frac{\dd x}{x}\, x^{s-1} \, f(x)
            \eqdelta
            \hat f(s)
            .
        \end{align}

        For all of the applications in this work, other than some derivations which are relegated to the Problems \sam{of Chapter...}, the functions in whose Mellin transforms we are interested will have support only on the interval \((0,1)\).
        %
        Therefore, unless stated otherwise, the Mellin transform appearing in this work will appear in the form
        \begin{align}
            \hat f(s)
            \eqdelta
            \int_0^1 \,\frac{\dd x}{x}\, x^{s-1} \, f(x)
            .
        \end{align}

        \sam{Put into the text where relevant}

        \sam{Mention here the part of the text where this appears and the problems in which it appears}

    \item
        \sam{LO, NLO, LL, NLL, etc.}

    \item
        The \textit{energy fraction} carried by a particular particle in a scattering event or jet, relative to the total energy of the event or jet, will play an important role in this thesis.
        %
        We will use the symbol \(z_i\) to denote the energy fraction particle \(i\) within a jet or event:
        \begin{align}
            z_i
            \eqdelta
            \frac{E_i}{E_\text{tot}}
            .
        \end{align}
        \sam{Revisit}

    \item
        We will also use the following standard definitions:
        \begin{subequations}
        \label{eq:qcd_coupling}
        \begin{align}
            \label{eq:as_defn}
            \as &= \frac{\alpha_s}{4\pi}
            \\
            \label{eq:beta_fn_defn}
            \frac{\dd}{\dd\log\mu}\alpha_s(\mu)
            &\eqdelta
            \beta(\alpha_s)
            =
            -2\beta_0 \alpha_s(\mu)^2 + \mathcal{O}\le(\alpha_s(\mu)^3\ri)
            \\
            \label{eq:alphas_1loop}
            \alpha_s^\text{(1-loop)}(\mu)
            &=
            \frac{\alpha_s(Q)}{1 + 2\alpha_s(Q) \beta_0 \log(\frac{\mu}{Q})}
            ,
        \end{align}
        where
        \begin{align}
            \beta_0 = \frac{1}{12\pi}\le(11\,C_A - 4\,T_f n_f\ri)
        \end{align}
        \end{subequations}
        is the leading coefficient of the QCD beta function, \(n_f = 5\) denotes the number of effectively massless quark flavors, \(Q\) indicates a reference scale (such as the energy scale of a hard process), and
        \begin{subequations}
        \label{eq:group_theory}
        \begin{align}
            C_F &= \frac{N_c^2 - 1}{2N_c}
            % \xrightarrow[]{\text{QCD}}
            % \frac{4}{3}
            \\
            C_A &= N_c
            % \xrightarrow[]{\text{QCD}}
            % 3
            \\
            T_F &= \frac{1}{2}
            ,
        \end{align}
        \end{subequations}
        denote the quadratic Casimir for the fundamental representation of \(SU(N_c)\), the quadratic Casimir for the adjoint representation, and the Dynkin index of the fundamental representation, respectively.
        %
        The expressions of \Eq{group_theory} are often referred to collectively as \textit{group theory factors}.
        %
        For QCD, \(N_c = 3\), \(C_F = 4/3\), and \(C_A = 3\), giving \(
            \beta_0 = \le(11 - 2 n_f / 3\ri)/4\pi
            .
        \)


%         A useful definition for clarity is that of the value of \(\alpha_s\) divided by \(4\pi\), since this quantity appears in loop computations (which often have a \(4\pi\) suppression):
%         \begin{subequations}
%         \begin{align}
%             a_s = \frac{\alpha_s}{4\pi}
%         \end{align}
%         \end{subequations}
\end{itemize}


Furthermore, we will assume
\begin{itemize}
    \item
    That all partons other than the top quark are massless;

    \item
    That we may ignore electroweak effects (i.e. perform calculations without the \(W\) and \(Z\) bosons).
\end{itemize}
