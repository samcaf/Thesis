% ==============================================
\chapter{Conclusions}
% ==============================================
\markboth{\small \textsc{Chapter \thechapter}: \bf Conclusions}{}
\label{sec:Conclusions}

\epigraph{
    \textit{Ignoramus et ignorabimus}

    [We do not know and we will not know]
}{
    Maxim on the limits of knowledge
}

\epigraph{
    \textit{Wir m\"ussen wissen -- wir werden wissen}.
    %
    [We must know -- we will know.]
}{
    \href{smith-at-sfsu.net/Documents/HilbertRadio/HilbertRadio.mp3}{David Hilbert, in response}
}

\epigraph{
    Do not condescend to ask:
    %
    ``Shall we conquer? Shall we be conquered?''
    %
    Fight on!
}{
    Nikos Kazantzakis
}

In this thesis, we studied perturbative manifestations and phenomenological probes of quantum chromodynamics (\gls{qcd}) in particle collisions.
%
After discussing the how quarks and gluons are the building blocks of the vast majority of our visible universe and the basics of \vocab{partonic scattering} in \Chap{particles} and the \vocab{partonic cascade model} for jet formation and substructure in \Chap{jets}, we realized that the real world presented greater challenges.
%
In particular, we understood that tests of our parton-level predictions of perturbative \gls{qcd} (\gls{pqcd}) require us to overcome the presence of \vocab{low-energy pollution}:
%
\gls{hadronization} and \gls{deteffects} (soft distortions) as well as the \glslink{ue}{underlying event} and \glslink{pileup}{pileup} (additive contamination).
%
With the tools of \gls{pqcd} honed in our introductory chapters, we undertook the task of \vocab{\gls{jet-grooming}} in \Chap{grooming}, where we explicitly removed soft radiation from jets in order to access high-energy, pollution-free degrees of freedom.
%
We began by building intuition for traditional hard-cutoff methods for grooming before building the framework of \PIRANHA{} for continuous grooming and showcasing its formal and phenomenological strengths.
%
Finally, in \Chap{ewocs}, we explored the techniques of \vocab{energy-weighted correlation functions}, with the goal of \textit{ignoring} the low-energy effects of pollution through energy weighting.
%
We introduced the basic theory of the energy-energy correlator (\gls{eec}) for probing angular correlations in collision events, discussed its efficient and visually intuitive higher point generalizations, and even initiated the study of \vocab{energy-weighted observable correlations} (\glspl{ewoc}) to probe arbitrary non-angular correlations, such as the pair-wise masses of subjets.


It has been a privilege to explore the universe with you.
%
Our goal has been to develop realistic tools for the study of the most fundamental features of our universe that humanity has been able to experimentally probe.
%
Our method was to develop collider observables whose behavior illuminates the structure of \gls{qcd} even through the obfuscating haze of contamination generated both by theoretical and experimental effects.
%
As stated by Andrew Larkoski, ``the art of jet substructure... is in the construction of observables'' \cite{}.
%
With the impressionist exposition and few new brush strokes presented in this thesis, we hope we have been able to share our sincere appreciation of the arts of jet substructure and quantum field theory.
%
A more precise understanding of the tools we presented here will require further, finer brush strokes:
%
precision calculations, experimental measurements, the modelling of theoretical and experimental uncertainties, and beyond.
%
But those tales are for another day, and another teller.
