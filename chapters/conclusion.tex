% ==============================================
\chapter{Conclusions}
% ==============================================
\markboth{\small \textsc{Chapter \thechapter}: \bf Conclusions}{}
\label{sec:Conclusions}

\epigraph{
    \textit{Ignoramus et ignorabimus}

    [We do not know and we will not know]
}{
    Maxim on the limits of knowledge
}

\epigraph{
    \textit{Wir m\"ussen wissen -- wir werden wissen}.

    [We must know -- we will know.]
}{
    \href{smith-at-sfsu.net/Documents/HilbertRadio/HilbertRadio.mp3}{David Hilbert, in response}
}

\epigraph{
    Do not condescend to ask:
    %
    ``Shall we conquer? Shall we be conquered?''
    %
    Fight on!
}{
    Nikos Kazantzakis
}

In this thesis, we studied perturbative manifestations and phenomenological probes of quantum chromodynamics (\gls{qcd}) in particle collisions.
%
After discussing the how quarks and gluons are the building blocks of the vast majority of our visible universe and the basics of \vocab{partonic scattering} in \Chap{particles} and the \vocab{partonic cascade model} for jet formation and substructure in \Chap{jets}, we realized that the real world presented greater challenges.
%
In particular, we understood that tests of our parton-level predictions of perturbative \gls{qcd} (\gls{pqcd}) require us to overcome the presence of \vocab{low-energy pollution}:
%
\gls{hadronization} and \gls{deteffects} (soft distortions) as well as the \glslink{ue}{underlying event} and \glslink{pileup}{pileup} (additive contamination).
%
With the tools of \gls{pqcd} honed in our introductory chapters, we undertook the task of \vocab{\gls{jet-grooming}} in \Chap{grooming}, where we explicitly removed soft radiation from jets in order to access high-energy, pollution-free degrees of freedom.
%
We began by building intuition for traditional hard-cutoff methods for grooming before building the framework of \PIRANHA{} for continuous grooming and showcasing its formal and phenomenological strengths.
%
Finally, in \Chap{ewocs}, we explored the techniques of \vocab{energy-weighted correlation functions}, with the goal of \textit{ignoring} the low-energy effects of pollution through energy weighting.
%
We introduced the basic theory of the energy-energy correlator (\gls{eec}) for probing angular correlations in collision events, discussed its efficient and visually intuitive higher point generalizations, and even initiated the study of \vocab{energy-weighted observable correlations} (\glspl{ewoc}) to probe arbitrary non-angular correlations, such as the pair-wise masses of subjets.


We've gained a precise, analytic understanding of some simple concepts in the phenomenological study of \gls{qcd} -- hopefully, at a level of detail that can accelerate your learning.
%
While it is \textit{always} helpful to have a solid understanding of the basics, the accuracy and utility of the results we discuss in this thesis can be improved by more detailed technical studies.
%
Precision calculations of \PIRANHA-groomed jet substructure, our new parametrization of energy correlators, and general \glspl{ewoc}, will enable more detailed comparisons of experimental and theoretical results, and lead to a deeper analytic understanding of the substructure observables we developed.
%
Additionally, the comparison of experiment and theory will need to be accompanied by more detailed modelling of uncertainties, such as the techniques of unfolding and re-weighting discussed in \App{pollution-models}.

More detailed studies of the mass \gls{ewoc}, for example, may lead to complementary measurements of masses of the particles of the standard model which decay hadronically.
%
The most precise existing measurements of the \(W\) boson mass involve leptonic decays of the \(W\) boson, rather than hadronic decays, and require a detailed understanding of recoil and missing energy;
%
the use of the mass \gls{ewoc} allows us to study the mass of the \(W\) boson with hadronic decays and without sensitivity to recoil%
\footnote{
    I thank Christoph Paus for pointing out this feature.
}
%
The study of the three-point mass \gls{ewoc} may also lead to a more precise measurement of the mass of the top quark, whose value sensitively determines the long-term fate of the standard model vacuum \cite{Alekhin:2012py}.

Furthermore, the substructure observables discussed in this thesis may be refined and honed for specific phenomenological tasks with more detailed optimization.
%
For example, we have already presented in \App{pufrenzy} a simple optimization of the \gls{rsf} \PIRANHA{} grooming procedure for the removal of pileup;
%
however, optimization of more general \glslink{rs}{recursive subtraction} algorithms using machine learning and more detailed recursive algorithms may lead to more robust \PIRANHA{} grooming techniques which are nonetheless tree-based, efficient, and connected to event-space geometry.
%
Optimization of the weights of \glslink{penc}{projected} and \glslink{renc}{resolved} energy correlators may allow us to perform interesting studies which classify the behavior of dense, high-multiplicity nuclear environments such as the quark-gluon plasma;
%
the efficient, tunable behavior of PENCs and RENCs may be used to test different models of QGP formation and scattering within the QGP.
%
Similarly, optimization of the weights of more general \glspl{ewoc} may enable even deeper understanding of the behavior of our microscopic universe which is more robust to the presence of additive contamination.
%
For example, we saw in the low-multiplicity environment of proton-proton collisions that mass \glspl{ewoc} which utilize higher energy weights lead to more robust mass extraction (see \Tab{energyweights} of \Sec{ewoc-mass}).
%
The high-multiplicity environment of heavy ion has an even more formidable underlying event of contaminating radiation whose effects can also be mitigated by the use of \glspl{ewoc} with higher energy weights.
%
Even further, the use of observables other than mass%
\footnote{
    For example, we mentioned in \Sec{ewoc-intro} that \glspl{ewoc} based on pairwise \textit{formation times} \(\tau_{ij} = \max{\le(E_i, E_j\ri)} / m^2_{ij}\) probe the time scales associated with the production of high energy structures within jets, and may be useful in probing the evolution of jets in heavy ion collisions.
}
%
may lead to additional power in classifying the behavior of the QGP and testing models of heavy ion collisions.
%
Optimizing the behavior of \glspl{ewoc} over both energy weights \textit{and} a wide space of pairwise observables (or even \(N\)-point observables) may lead to \glspl{ewoc} which discriminate sharply between different models of the QGP, and of particle production in high-energy scattering events in general.


It has been a privilege to explore the universe with you.
%
Our goal has been to develop realistic tools for the study of the most fundamental features of our universe that humanity has been able to experimentally probe.
%
Our method was to develop collider observables whose behavior illuminates the utructure of \gls{qcd} even through the obfuscating haze of contamination generated both by theoretical and experimental effects.
%
As stated by Andrew Larkoski, ``the art of jet substructure... is in the construction of observables'' \cite{Larkoski:2024uoc}.
%
With the impressionist exposition and few new brush strokes presented in this thesis, we hope we have been able to share our sincere appreciation of the arts of jet substructure and quantum field theory.
%
A more precise understanding of the tools we presented here will require further, finer brushwork:
%
precision calculations, experimental measurements, the modelling of theoretical and experimental uncertainties, and beyond.
%
But those tales are for another day, and another teller.
