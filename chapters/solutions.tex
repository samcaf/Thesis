% ==============================================
\section*{Chapter 2}
% ==============================================

% ==============================================
\section*{Chapter 3}
% ==============================================

\sam{FIGURE OUT LINKS}


% ==============================================
\section*{Bonus Problems}
% ==============================================

% ---------------------------------------
\BonusSoln{mellinconvolution}
% ---------------------------------------
\sam{FIX BOUNDS}

We want to show that
\begin{align*}
    \mc M[f \mellinconvolution g](s) = \mc M[f](s) \mc M[g](s)
    ,
\end{align*}
where
\(
    \mathcal{M}[f](s) = \int_0^\infty \dd x\, x^{s-1} f(x)
    ,
\)
and
\(
    (f(x) \mellinconvolution g(x))(x)
    =
    \int_x^1 \frac{\dd y}{y}\, f(y) g(x/y)
    .
\)


To solve the problem, it is helpful to re-organize the integral associated with the Mellin transform of the convolution of \(f(x)\) and \(g(x)\):
\begin{align*}
    \int_0^\infty \dd x\, x^{s-1}
    \int_0^\infty \,\frac{\dd y}{y}\,
    f(y) g(x/y)
    =
    \int_0^\infty \,\frac{\dd y}{y}
    f(y)
    \int_0^\infty \dd x\,
    x^{s-1} g(x/y)
    =
    \int_0^1 \dd y \, y^{s-1} f(y)
    \int_0^1 \dd z\, z^{s-1} g(z)
    =
    f(s) g(s)
    .
\end{align*}
In the first equality, we have exchanged the two integrals. % and noted that \(x \in (0,1),\, y \in (x,1)\) cuts out the same region of the plane as \(y \in (0, 1),\,x \in (0,y)\). % From when the bounds were from zero to one
%
In the second equality, we have changed variables from \(x\) to \(z = x/y\) and used \(\dd x\, x^n = y^{n+1}\, \dd z\, z^n\).

\qed{}

\sam{Add discussion of what happens when bounds are from 0 to 1}


% ---------------------------------------
\BonusSoln{mellinconvolution2}
% ---------------------------------------
We want to show that
\begin{align*}
    \mc M[f \circ_{\mc M} g](s) = \mc M[f](s) \mc M[g](s)
    ,
\end{align*}
where
\(
    \mathcal{M}[f](s) = \int_0^\infty \,\dd x\, x^{s-1} f(x)
    ,
\)
and
\(
    (f(x) \circ_{\mc M} g(x))(x)
    =
    \int_0^\infty\,\dd\xi f(x\,\xi) g(\xi)
    .
\)


To solve the problem, it is helpful to re-organize the integral associated with the Mellin transform of the convolution of \(f(x)\) and \(g(x)\):
\begin{align*}
    \int_0^\infty \dd x\, x^{s-1}
    \int_0^\infty \, \dd \xi
    f(\xi x) g(\xi)
    =
    \int_0^\infty \, \dd \xi
    \int_0^\infty \dd (\xi x)\, (\xi x)^{s-1}
    \frac{1}{\xi^s}
    f(\xi x) g(\xi)
    =
    \int_0^\infty \, \dd \xi
    \xi^{-s}
    g(\xi)
    \int_0^\infty \dd z\, z^{s-1}
    f(z)
    =
    f(s) g(1-s)
    .
\end{align*}
In the first equality, we have exchanged the two integrals and changed variables from \(x\) to \(\xi x = z\), and in the final equality we have used \(-s = (1-s) - 1\) to note that the first integral is equal to \(g(1-s)\).

\qed{}

% ---------------------------------------
\BonusSoln{abstractmellin}
% ---------------------------------------

\begin{align}
%\label{eq:}
    x^b f(x)
    &\to
    \\
    \frac{\dd}{\dd x} f(x)
    &\to
    \\
    \int_0^x f(t) \dd t
    &\to
    \\
    f(x^a)
    &\to
    \frac{1}{a} \hat f(s/a)
\end{align}

\sam{intuition ffor each}





% ---------------------------------------
\BonusSoln{concretemellin}
% ---------------------------------------


\begin{align}
%\label{eq:}
    x^n
    &\to
    \\
    \frac{1}{1+x}
    &\to
    \beta(s, 1-s)
    =
    \Gamma(s) \Gamma(1-s)
    \\
    e^{-x}
    &\to
    \Gamma(s)
    \\
    e^{-a x}
    &\to
    \Gamma(s) a^{-s}
    \\
    \sin(k x)
    &\to
    \Gamma(s) \sin\left(\frac{\pi s}{2}\right) k^{-s}
    \\
    \cos(k x)
    &\to
    \Gamma(s) \cos\left(\frac{\pi s}{2}\right) k^{-s}
    \\
    e^{-x^2}
    &\to
    \frac{1}{2}\Gamma\le(\frac{s}{2}\ri)
    \\
    \frac{1}{e^{x} - 1}
    &\to
    \Gamma(s)\zeta(s)
    \\
    \frac{1}{e^{x} + 1}
    &\to
    \le(1 - 2^{1-s}\ri)\Gamma(s)\zeta(s)
\end{align}


% ==============================================
\section*{Chapter 4}
% ==============================================


% ==============================================
\section*{Chapter 5}
% ==============================================
