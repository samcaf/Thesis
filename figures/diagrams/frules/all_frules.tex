\begin{answer}
\begin{center}
{\normalfont\Large\bfseries\sffamily The Rules for the Gluon Sector of QCD are:}
\end{center}

\begin{subequations}
\begin{align}
\raisebox{0pt}{
\begin{tikzpicture}
    \begin{feynman}
        % vertices
        \vertex (i) at (-1.5,0.0);
        \vertex (f) at (1.5,0.0);
        % edge labels
        \vertex [above left=-0.8pt and 0.0ptof i] (ti) {$a, \mu$};
        \vertex [above right=0.0pt and 0.0pt of f] (tf) {$b,\nu$};
        % diagram
        \diagram* {
            (i)
            -- [gluon, momentum=\(p\)]
            (f)
            ,
        };
    \end{feynman}
\end{tikzpicture}
}
\quad
:=
\quad
&
i\, \delta^{ab} \frac{-g_{\mu\nu} + (1-\xi) p_\mu p_\nu / p^2}{p^2 + i 0^+}
\\
\notag
\quad
\xrightarrow[\text{Feynman gauge}]{}
\quad
&
\delta^{ab} \frac{-i g_{\mu\nu}}{p^2 + i 0^+}
\end{align}

\begin{align}
\raisebox{-40pt}{
\begin{tikzpicture}
    \begin{feynman}
        % vertices
        \vertex (u) at (0,1.5);
        \vertex (l) at (-1.30,-0.75);
        \vertex (r) at (1.30,-0.75);
        \vertex (cen) at (0,0);
        % edge labels
        \vertex [below left=-0.8pt and 0.0ptof l] (tl) {$a, \mu$};
        \vertex [above left=-0.8pt and 0.0ptof u] (tu) {$b, \nu$};
        \vertex [above right=0.0pt and 0.0pt of r] (tr) {$c,\rho$};
        % diagram
        \diagram* {
            (l)
            -- [gluon, momentum=\(p\)]
            (cen)
            ,
            (u)
            -- [gluon, momentum=\(q\)]
            (cen)
            ,
            (r)
            -- [gluon, momentum=\(r\)]
            (cen)
        };
    \end{feynman}
\end{tikzpicture}
}
\quad
:=
\quad
-g f^{abc} \le[
    \begin{array}{rrr}
    (q-r)_\mu
    &
    g_{\nu\rho}
    &
    \\
    +&&
    \\
    (r-p)_\nu
    &
    g_{\rho\mu}
    &
    \\
    +&&
    \\
    (p-q)_\rho
    &
    g_{\mu\nu}
    &
    \end{array}
\ri]
\,.
\end{align}

\begin{align}
\raisebox{-40pt}{
\begin{tikzpicture}
    \begin{feynman}
        % vertices
        \vertex (ul) at (-1.0,1.0);
        \vertex (ll) at (-1.0,-1.0);
        \vertex (ur) at (1.0,1.0);
        \vertex (lr) at (1.0,-1.0);
        \vertex (cen) at (0,0);
        % edge labels
        \vertex [above left=-0.8pt and 0.0ptof ul] (tu) {$a, \mu$};
        \vertex [above left=-0.8pt and 0.0ptof ll] (tl) {$b, \nu$};
        \vertex [above right=0.0pt and 0.0pt of ur] (tr) {$c,\rho$};
        \vertex [above right=0.0pt and 0.0pt of lr] (tr) {$d,\sigma$};
        % diagram
        \diagram*{
            (ul)
            -- [gluon]
            (cen)
            ,
            (ur)
            -- [gluon]
            (cen)
            ,
            (ll)
            -- [gluon]
            (cen)
            ,
            (lr)
            -- [gluon]
            (cen)
        };
    \end{feynman}
\end{tikzpicture}
}
\quad
:=
\quad
&
-i g^2
\le[
\begin{array}{rrr}
    & f^{\bar{e}ac} f^{\bar{e}bd} &\le(g_{\mu\nu} g_{\rho\sigma} - g_{\mu\sigma} g_{\nu\rho}\ri)
    \\
    +& f^{\bar{e}ad} f^{\bar{e}bc} &\le(g_{\mu\nu} g_{\rho\sigma} - g_{\mu\rho} g_{\nu\sigma}\ri)
    \\
    +& f^{\bar{e}ab} f^{\bar{e}cd} &\le(g_{\mu\rho} g_{\nu\sigma} - g_{\mu\sigma} g_{\nu\rho}\ri)
\end{array}
\ri]
\,,
\end{align}
where the ``barred'' index \(\bar{e}\) is marked only to indicate that it is summed over.

\end{subequations}
with the counterterm diagrams
\begin{subequations}
% \input{}
\end{subequations}
\end{answer}

\begin{answer}
\begin{center}
{\normalfont\Large\bfseries\sffamily The Rules for the Quark Sector of QCD are:}
\end{center}

\begin{subequations}
\begin{align}
\raisebox{0pt}{
\begin{tikzpicture}
    \begin{feynman}
        % vertices
        \vertex (i) at (-1.5,0.0);
        \vertex (f) at (1.5,0.0);
        % edge labels
        \vertex [above left=-0.8pt and 0.0ptof i] (ti) {$i$};
        \vertex [above right=0.0pt and 0.0pt of f] (tf) {$j$};
        % diagram
        \diagram* {
            (i)
            -- [fermion, momentum=\(p\)]
            (f)
            ,
        };
    \end{feynman}
\end{tikzpicture}
}
\quad
:=
\quad
&
i\, \delta\indices{^j_i} \frac{\slashed{p} + m}{p^2 - m^2 + i 0^+}
\end{align}

\begin{align}
\raisebox{-40pt}{
\begin{tikzpicture}
    \begin{feynman}
        % vertices
        \vertex (u) at (0,1.5);
        \vertex (l) at (-1.30,-0.75);
        \vertex (r) at (1.30,-0.75);
        \vertex (cen) at (0,0);
        % edge labels
        \vertex [above right=-0.8pt and 0.0ptof u] (tu) {$a, \mu$};
        \vertex [below left=-0.8pt and 0.0ptof l] (tl) {$i$};
        \vertex [below right=0.0pt and 0.0pt of r] (tr) {$j$};
        % diagram
        \diagram* {
            (cen)
            -- [gluon]%, momentum'=\(q\)]
            (u)
            ,
            (l)
            -- [fermion]%, momentum=\(p\)]
            (cen)
            ,
            (cen)
            -- [fermion]%, momentum=\(p-q\)]
            (r)
        };
    \end{feynman}
\end{tikzpicture}
}
\quad
:=
\quad
-i g \le(T^a\ri)\indices{^j_i}
\gamma^\mu
\,.
\end{align}

\end{subequations}
with the counterterm diagrams
\begin{subequations}
% \input{}
\end{subequations}
Let's also not forget that any time we have a closed fermion loop, the associated diagram gets an additional factor of \(-1\).
\end{answer}

\begin{answer}
\begin{center}
{\normalfont\Large\bfseries\sffamily The Rules for the Ghost Sector of QCD are:}

(in covariant gauges)
\end{center}

\begin{subequations}
\begin{align}
\raisebox{0pt}{
\begin{tikzpicture}
    \begin{feynman}
        % vertices
        \vertex (i) at (-1.5,0.0);
        \vertex (f) at (1.5,0.0);
        % edge labels
        \vertex [above left=-0.8pt and 0.0ptof i] (ti) {$a$};
        \vertex [above right=0.0pt and 0.0pt of f] (tf) {$b$};
        % diagram
        \diagram* {
            (i)
            -- [dotted, ultra thick, momentum=\(p\)]
            (f)
            ,
        };
    \end{feynman}
\end{tikzpicture}
}
\quad
:=
\quad
&
\delta^{ab} \frac{i}{p^2 + i 0^+}
\end{align}

\begin{align}
\raisebox{-40pt}{
\begin{tikzpicture}
    \begin{feynman}
        % vertices
        \vertex (u) at (0,1.5);
        \vertex (il) at (-0.65,-0.375);
        \vertex (l) at (-1.30,-0.75);
        \vertex (ir) at (0.65,-0.375);
        \vertex (r) at (1.30,-0.75);
        \vertex (cen) at (0,0);
        % edge labels
        \vertex [above right=-0.8pt and 0.0ptof u] (tu) {$a, \mu$};
        \vertex [below left=-0.8pt and 0.0ptof l] (tl) {$b$};
        \vertex [below right=0.0pt and 0.0pt of r] (tr) {$c$};
        % diagram
        \diagram* {
            (cen)
            -- [gluon]%, momentum'=\(q\)]
            (u)
            ,
            (l)
            -- [dotted, arrow, ultra thick]
            (il)
            -- [dotted, ultra thick]
            (cen)
            ,
            (cen) -- [dotted, ultra thick, momentum=\(r\)] (r)
            ,
            (cen)
            -- [dotted, arrow, ultra thick]
            (ir)
            -- [dotted, ultra thick]
            (r)
            ,
        };
    \end{feynman}
\end{tikzpicture}
}
\quad
:=
\quad
g f^{abc} r^\mu
\,.
\end{align}

\end{subequations}
with the counterterm diagrams
\begin{subequations}
% \input{}
\end{subequations}
\end{answer}
